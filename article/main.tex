
% \documentclass[preprint,12pt]{elsarticle}
%% Use the option review to obtain double line spacing
\documentclass[preprint,review,12pt]{elsarticle}

%% Use the options 1p,twocolumn; 3p; 3p,twocolumn; 5p; or 5p,twocolumn
%% for a journal layout:
%% \documentclass[final,1p,times]{elsarticle}
%% \documentclass[final,1p,times,twocolumn]{elsarticle}
%% \documentclass[final,3p,times]{elsarticle}
%% \documentclass[final,3p,times,twocolumn]{elsarticle}
%% \documentclass[final,5p,times]{elsarticle}
%% \documentclass[final,5p,times,twocolumn]{elsarticle}

\journal{Information Sciences}

\usepackage{acro}

% Feedforward Neural Network
\DeclareAcronym{FFNN}{
      short = FFNN,
      short-plural-form = FFNNs,
      long = \textit{feedforward neural network},
      long-plural-form = \textit{feedforward neural networks}
}

% Hyper-Heuristic
\DeclareAcronym{HH}{
      short = HH,
      short-plural-form = HHs,
      long = \textit{hyper-heuristic},
      long-plural-form = \textit{hyper-heuristics}
}

% Bayesian Hyper-Heuristic
\DeclareAcronym{BHH}{
      short = BHH,
      short-plural-form = BHHs,
      long = \textit{Bayesian hyper-heuristic},
      long-plural-form = \textit{Bayesian hyper-heuristics}
}

\begin{document}

\begin{frontmatter}
  %% Title, authors and addresses

  %% use the tnoteref command within \title for footnotes;
  %% use the tnotetext command for theassociated footnote;
  %% use the fnref command within \author or \address for footnotes;
  %% use the fntext command for theassociated footnote;
  %% use the corref command within \author for corresponding author footnotes;
  %% use the cortext command for theassociated footnote;
  %% use the ead command for the email address,
  %% and the form \ead[url] for the home page:
  %% \title{Title\tnoteref{label1}}
  %% \tnotetext[label1]{}
  %% \author{Name\corref{cor1}\fnref{label2}}
  %% \ead{email address}
  %% \ead[url]{home page}
  %% \fntext[label2]{}
  %% \cortext[cor1]{}
  %% \affiliation{organization={},
  %%             addressline={},
  %%             city={},
  %%             postcode={},
  %%             state={},
  %%             country={}}
  %% \fntext[label3]{}

  \title{Bayesian Hyper-Heuristics}

  %% use optional labels to link authors explicitly to addresses:
  \author[aff:tuks]{A.N.~Schreuder}
  \author[aff:stellenbosch]{Prof.~A.P.~Engelbrecht}
  \author[aff:tuks]{Dr.~A.S.~Bosman}
  \author[aff:wits]{Prof.~C.W.~Cleghorn}

  \affiliation[aff:tuks]{
    organization={University of Pretoria},
    % addressline={Lynnwood Rd},
    city={Pretoria},
    postcode={0002},
    state={Gauteng},
    country={South Africa}
  }

  \affiliation[aff:stellenbosch]{
    organization={Stellenbosch University},
    % addressline={Ryneveld Street},
    city={Stellenbosch},
    postcode={7602},
    state={Western Cape},
    country={South Africa}
  }

  \affiliation[aff:wits]{
    organization={University of the Witwatersrand},
    % addressline={1 Jan Smuts Ave},
    city={Johannesburg},
    postcode={2000},
    state={Gauteng},
    country={South Africa}
  }

  \begin{abstract}

    This work introduces a novel population-based \Acf{BHH}. An empirical study is done by using the \acs{BHH} to train \acsp{FFNN}. An in-depth behaviour analysis is done and the performance of the \acs{BHH} is compared to that of ten popular low-level heuristics, each with different search behaviours. The chosen heuristic pool consists of classic gradient-based heuristics as well as meta-heuristics. The empirical process is executed on fourteen datasets consisting of classification and regression problems with varying characteristics. The \acs{BHH} is shown to be able to train \acsp{FFNN} well and provide an automated method for finding the best heuristic to train the \acsp{FFNN} at various stages of the training process.

  \end{abstract}

  %%Graphical abstract
  % \begin{graphicalabstract}
  % \includegraphics{grabs}
  % \end{graphicalabstract}

  %%Research highlights
  \begin{highlights}
    \item A novel heuristic selection operator is used that focuses on using Bayesian statistics to calculate the probability that a heuristic should be selected in order to efficiently train \acp{FFNN}. The resulting \acs{HH} is referred to as the \acf{BHH}.

    \item The results of the empirical study show with statistical significance and certainty that the \Acs{BHH} performs generally well on multiple problems. It is shown that, for each problem, the \Acs{BHH} performance is comparable to the best low-level heuristics included in the heuristic selection pool.

    \item The results of the empirical study show that the \Acs{BHH} is able to select the best heuristic to train \acp{FFNN} in general. This relieves researchers from the burden of having to do this selection process manually through trial and error.

    \item The results of the empirical study show that the \Acs{BHH}, given a diverse set of lower-level heuristics, will generally produce good results when applied to multiple problems at the same time.

    \item Finally, the results of the empirical study show that the \Acs{BHH} is capable of utilising \textit{a priori} \footnote{Latin word, meaning ``from what comes before''.} knowledge in which a predefined selection bias is used for heuristics that are known to be well suited for certain problems.
  \end{highlights}

  \begin{keyword}

    hyper-heuristics \sep meta-learning \sep feedforward neural networks \sep supervised learning \sep Bayesian statistics

  \end{keyword}

\end{frontmatter}

\section{Introduction}
\label{sec:introduction}

\section{Artifical Neural Networks}
\label{sec:anns}

\section{Heuristics}
\label{sec:heuristics}

\section{Hyper-Heuristics}
\label{sec:hhs}

\section{Probability}
\label{sec:probability}

\section{Bayesian Hyper-Heuristics}
\label{sec:bhh}

\section{Methodology}
\label{sec:methodology}

\section{Results}
\label{sec:results}

\section{Conclusion}
\label{sec:conclusion}

\appendix

\section{Appendix A}
\label{app:appendix_a}

\section{Appendix B}
\label{app:appendix_b}

%% For citations use: 
%%       \citet{<label>} ==> Jones et al. [21]
%%       \citep{<label>} ==> [21]
%%

%% If you have bibdatabase file and want bibtex to generate the
%% bibitems, please use
%%
%%  \bibliographystyle{elsarticle-num-names} 
%%  \bibliography{<your bibdatabase>}

%% else use the following coding to input the bibitems directly in the
%% TeX file.

\begin{thebibliography}{00}

  %% \bibitem[Author(year)]{label}
  %% Text of bibliographic item

  \bibitem[ ()]{}

\end{thebibliography}
\end{document}

\endinput
