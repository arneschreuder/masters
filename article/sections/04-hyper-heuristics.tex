\section{Hyper-Heuristics}
\label{sec:hhs}

\citeauthor{ref:burke:2010} \cite{ref:burke:2010} define \acp{HH} as search methods or learning mechanism for selecting or generating heuristics to solve computational search problems. \citeauthor{ref:burke:2003} \cite{ref:burke:2003} mentions that a \acs{HH} is a high-level heuristic approach that, given a particular problem instance and a number of low-level heuristics, can select and apply an appropriate low-level heuristic at each decision point. \acp{HH} implement a form of \textit{meta-learning} that is concerned with the selection of the best heuristic from a pool of heuristics to solve a given problem. It can be said that \acp{HH} are concerned with finding the best heuristic in \textit{heuristic space}, while the underlying low-level heuristics find solutions in the feasible \textit{search/solution space}.

\citeauthor{ref:burke:2010} \cite{ref:burke:2010} proposed a modern classification scheme used to classify \acp{HH}. According to the proposed classification scheme, \acp{HH} are classified in two dimensions. These include the \textit{source of feedback} used during learning and the nature of the \textit{heuristic search space}. For the dimension that involves the source of feedback, \acp{HH} can be classified as either \textit{no learning}, \textit{online learning} or \textit{offline learning}. For the dimension that involves the nature of the \textit{heuristic search space}, \acp{HH} can be classified as either \textit{heuristic selection} or \textit{heuristic generation}. Further distinction is made between \textit{construction} of heuristics and \textit{perturbation} of heuristics.

In the general context of optimisation, many different types of \acp{HH} have been implemented and applied to many different problems. Some notable examples include \cite{ref:dowsland:2007, ref:burke:2010, ref:grobler:2012, ref:vanderstockt:2018}. Research on the application of \acp{HH} in the context of \acs{FFNN} training is still scarce. \citeauthor{ref:nel:2021} \cite{ref:nel:2021} provides some of the first research in this field, applying a \acs{HH} to \acs{FFNN} training.

This research takes a particular interest in a population-based, selection approach for \acp{HH}, with the particular intent of training \acp{FFNN}. In the context of population-based \acp{HH}, an entity pool exists that represent a pool of candidate solutions to the given problem. Each entity in the entity pool is assigned its own low-level heuristic from the heuristic pool. The selection of the best heuristic to apply to a candidate solution, is based on the performance of the heuristic relative to that particular candidate solution at a particular point in the search process. Selection methods often make use of probabilistic approaches.