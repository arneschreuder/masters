\pagestyle{empty}

\begin{center}
    \Large
    \textbf{Training Feedforward Neural Networks with Bayesian Hyper-Heuristics}
    \vspace{0.5cm}

    \normalsize
    by
    \vspace{0.5cm}

    \normalsize
    A.N. Schreuder\\
    \ifpdf
        E-mail: \href{mailto:arne@schreuder.ai}{arne@schreuder.ai}
    \else
        E-mail: arne@schreuder.ai
    \fi
    \vspace{1cm}

    \Large
    \textbf{Abstract}
\end{center}

Many different heuristics have been developed and used to train feedforward
neural networks (FFNNs).  However, selection of the best heuristic to train
FFNNs is a time consuming and non-trivial exercise. Careful, systematic
selection is thus required to select the best heuristic. In the past, selection
was done by trial and error. A modern approach is to automate this selection
proses. However, a single approach might not be sufficient. Research has
proposed the use of hybridisation of heuristics. These methods are referred to
as hyper-heuristics (HH) and focus on finding the best heuristic in
heuristic-space by making use of heuristic performance information. One such
form of HH is a population-based approach that guides the search process by
automatically selecting heuristics from a heuristic-pool to be applied to
different entities that represent candidate solutions to the problem-space and
work together to find good solutions. This thesis introduces a novel Bayesian
hyper-heuristic (BHH) to train FFNNs.  Bayesian analysis is used to guide the
selection of lower-level heuristics by updating prior selection probability
distributions with performance biased posteriors. An empirical study is done by
using the BHH to train FFNNs. The performance of the BHH is compared to that of
\todo[inline]{get final number}eight classic heuristics with different search
behaviours.  The emperical test was executed on \todo[inline]{get final
number}ten datasets consisting out of classification and regression problems
with varying characteristics. The BHH is shown to be able to train FFNNs and
provide an automated method for finding the best heuristic to train the
\ac{FFNN}.\\*[1cm]
\noindent
\parbox{\textwidth}{
    \textbf{Keywords:} hyper-heuristics, meta-learning, feedforward neural
    networks, supervised learning, optimisation.
}
\vfill
\newpage

