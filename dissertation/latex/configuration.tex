% Biblatex
\addbibresource{./bibliography/references.bib}

% Acro
\acsetup {
    make-links=true
}

% Fancy headers and footers
\fancyhead[L]{\textsf{\nouppercase{\leftmark}}} % Left side of header
\fancyhead[R]{\textsf{\thepage}} % Right side of header
\fancyfoot{} % No footer
\renewcommand{\headrulewidth}{1pt} % Line under header

% Graphics
\graphicspath{{./images/}{./plots/}{./analysis/}} % Include path for images

% % Hyperlinks
\hypersetup{
    breaklinks=true,   %  Allow link breaking across lines
    citecolor=blue,    %  Citation link color
    colorlinks=true,   %  Color link text only (no borders)
    filecolor=blue,    %  File link color
    linkcolor=blue,    %  Internal link color
    menucolor=blue,    %  Acrobat menu link color
    pageanchor=true,   %  Enable page referencing
    pdffitwindow=true, %  Fit pdf to window
    pdfhighlight=/N,   %  No visual cue on clicking link
    pdfstartview=Fit,  %  Fit initial view to page
    plainpages=false,  %  Prevent hyperref page number changes
    urlcolor=blue,      %  URL link color,
    draft=false 
}

% PDF Meta
\hypersetup{
    pdftitle    = {Training Feedforward Neural Networks with Bayesian
    Hyper-Heuristics},
    pdfauthor   = {A.N. Schreuder <arne@schreuder.ai>},
    pdfsubject  = {Bayesian Hyper-Heuristics},
    pdfkeywords = {Hyper-Heuristics, Meta-Learning, Feedforward Neural
    Networks, Supervised Learning, Bayesian Statistics, Optimisation.},
    % draft              %  Remove on final document
}

% Dates
\newdateformat{monthyeardate}{\monthname[\THEMONTH], \THEYEAR}

% Force LaTeX-compliant spacing
\pdfadjustspacing=1 

% Sets minimum compatibility for pdfplotset
\pgfplotsset{compat=1.17} 

% Lengths
\setlength{\headheight}{15pt}
\setlength{\marginparwidth}{2cm}

% Line spacing
\linespread{1.3}

% URLs
\urlstyle{same} % Set the URL font to the same as the text font

% Define graph float
\newfloat{graph}{tbp}{lgf}[chapter]
\floatname{graph}{Graph}

% Define algorithm float
\newfloat{algorithm}{tbp}{loa}
\floatname{algorithm}{Algorithm}

% Bold float caption numbers and reduced size captions
\makeatletter
\long\def\@makecaption#1#2{%
    \vskip\abovecaptionskip
    \sbox\@tempboxa{{\small{\bf #1:} #2}}%
    \ifdim \wd\@tempboxa >\hsize
        {\small{\bf #1:} #2\par}
    \else
        \hbox to\hsize{\hfil\box\@tempboxa\hfil}%
    \fi
    \vskip\belowcaptionskip
}
\renewcommand\floatc@plain[2]{
    \setbox\@tempboxa\hbox{\small{\bf #1:} #2}%
    \ifdim\wd\@tempboxa>\hsize
        {\small{\bf #1:} #2\par}
    \else
        \hbox to\hsize{\hfil\box\@tempboxa\hfil}
    \fi
}
\makeatother

% Declare argmin and argmax math operators
\DeclareMathOperator*{\argmin}{arg\,min}
\DeclareMathOperator*{\argmax}{arg\,max}

\makeatletter
\g@addto@macro\normalsize{%
  \setlength\abovedisplayskip{5pt}
  \setlength\belowdisplayskip{20pt}
  \setlength\abovedisplayshortskip{5pt}
  \setlength\belowdisplayshortskip{20pt}
}
\makeatother

% Defines a theorem keyword
\newtheorem{theorem}{Theorem}

% Defines a definition keyword
\newtheorem{definition}{Definition}[subsection]

% Defines a example keyword
\newtheorem{example}{Example}
