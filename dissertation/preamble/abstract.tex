\pagestyle{empty}

\begin{center}
    \Large
    \textbf{Training Feedforward Neural Networks with Bayesian Hyper-Heuristics}
    \vspace{0.5cm}

    \normalsize
    by
    \vspace{0.5cm}

    \normalsize
    A.N. Schreuder\\
    \ifpdf
        E-mail: \href{mailto:an.schreuder@up.ac.za}{an.schreuder@up.ac.za}
    \else
        E-mail: an.schreuder@up.ac.za
    \fi
    \vspace{1cm}

    \Large
    \textbf{Abstract}
\end{center}

Many different heuristics have been developed and used to train \acfp{FFNN}.  However, selection of the best heuristic to train
\acsp{FFNN} is a time consuming and non-trivial exercise. Careful, systematic
selection is thus required to ensure that the best heuristic is used to train the \acsp{FFNN}. In the past, selection
was done by trial and error. A modern approach is to automate the heuristic selection
process. Often it is found that a single approach is not sufficient. Research has
proposed the use of hybridisation of heuristics. One such approach is referred to
as \acfp{HH}. \acsp{HH} focus on dynamically finding the best heuristic or combinations of heuristics in
heuristic-space by making use of heuristic performance information during training time. One such
implementation of a \acs{HH} is a population-based approach that guides the search process by
dynamically selecting heuristics from a heuristic-pool to be applied to
different entities that represent candidate solutions to the problem-space and
work together to find good solutions. This dissertation introduces a novel population-based \Acf{BHH}. An empirical study is done by
using the \acs{BHH} to train \acsp{FFNN}. An in-depth behaviour analysis is done and the performance of the \acs{BHH} is compared to that of ten popular low-level heuristics each with different search
behaviours. The chosen heuristic pool consists out of classic gradient-based heuristics as well as meta-heuristics. The empirical process was executed on fifteen datasets consisting of classification and regression problems
with varying characteristics. Results are analysed for statistical significance and the \acs{BHH} is shown to be able to train \acsp{FFNN} well and provide an automated method for finding the best heuristic to train the \acsp{FFNN} at various stages of the training process.\\*[1cm]
\noindent
\parbox{\textwidth}{
    \textbf{Keywords:} hyper-heuristics, meta-learning, feedforward neural
    networks, supervised learning, Bayesian statistics.
}
\vfill
\newpage

