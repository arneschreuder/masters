\chapter{Results}
\label{chap:results}

\begin{quotation}
      \noindent ``Without data, you are just another person with an opinion.''
\end{quotation}
\begin{flushright}
      - W. Edwards Deming
\end{flushright}

\noindent
All the required background information on all the elements relevant to the empirical process have been presented in Chapters~\ref{chap:anns}-\ref{chap:probability}. The proposed \acf{BHH} was presented in Chapter~\ref{chap:bhh} and the design and methodology of the empirical process was presented in Chapter~\ref{chap:methodology}. The methodology presented a number of experiments to conduct, including a case study on the behaviour of the \acs{BHH} during training, and a comparison to the performance of state-of-the-art, standalone, low-level heuristics. The methodology also defined a number of experiments related to the empirical testing of the effects of hyper-parameters on the outcomes of the \acs{BHH}.

Finally, this chapter provides the outcome of these empirical processes and provides results on all the experiments that have been conducted. Detailed discussions follow on the outcomes of each experiment. Discussions are accompanied by figures and plots to help provide visual aid for the discussions. All experiments and discussions also provide the outcome of the statistical analysis of the results. The remainder of the chapter is structured as follows.

\begin{itemize}
      \item \textbf{Section \ref{sec:results:case_study}} provides detailed discussions on the outcomes of the case study on the behaviour of the \acs{BHH}. Illustrations are provided to show the outcomes of the learning process of the \acs{BHH}, while training the underlying \acf{FFNN}.

      \item \textbf{Section \ref{sec:results:bhh_vs_low_level_heuristics}} provides the results of the performance of the \acs{BHH} compared to individual low-level heuristics as it relates to all datasets. Three variants of the \acs{BHH} are included in these results. These include the baseline configuration, as well as a configuration that only includes gradient-based heuristics in the \index{heuristic pool}heuristic pool, and a configuration that only includes meta-heuristics in the \index{heuristic pool}heuristic pool.

      \item \textbf{Section \ref{sec:results:bhh_variant_hp}} provides the results for the experimental group that analyses the effects of the \index{heuristic pool}\textit{heuristic pool} hyper-parameter on the outcomes of the \acs{BHH}.

      \item \textbf{Section \ref{sec:results:bhh_variant_population}} provides the results for the experimental group that analyses the effects of the \textit{population size} hyper-parameter on the outcomes of the \acs{BHH}.

      \item \textbf{Section \ref{sec:results:bhh_variant_burn_in}} provides the results for the experimental group that analyses the effects of the \textit{burn in window size} hyper-parameter on the outcomes of the \acs{BHH}.

      \item \textbf{Section \ref{sec:results:bhh_variant_credit}} provides the results for the experimental group that analyses the effects of the \textit{credit assignment strategy} hyper-parameter on the outcomes of the \acs{BHH}.

      \item \textbf{Section \ref{sec:results:bhh_variant_replay}} provides the results for the experimental group that analyses the effects of the \textit{replay window size} hyper-parameter on the outcomes of the \acs{BHH}.

      \item \textbf{Section \ref{sec:results:bhh_variant_reanalysis}} provides the results for the experimental group that analyses the effects of the \textit{reanalysis interval} hyper-parameter on the outcomes of the \acs{BHH}.

      \item \textbf{Section \ref{sec:results:bhh_variant_reselection}} provides the results for the experimental group that analyses the effects of the \textit{reselection interval} hyper-parameter on the outcomes of the \acs{BHH}.

      \item \textbf{Section \ref{sec:results:bhh_variant_normalise}} provides the results for the experimental group that analyses the effects of the \textit{normalisation} hyper-parameter on the outcomes of the \acs{BHH}.

      \item \textbf{Section \ref{sec:results:bhh_variant_discounted_rewards}} provides the results for the experimental group that analyses the effects of the \textit{discounted rewards} hyper-parameter on the outcomes of the \acs{BHH}.

      \item \textbf{Section \ref{sec:results:overfitting}} provides a brief discussion on overfitting as it is observed for some of the empirical tests.

      \item \textbf{Section \ref{sec:results:computational_req}} provides a brief discussion on the practical application of the \acs{BHH} to real-world problems.

      \item \textbf{Section \ref{sec:results:overview}} provides an overview of all the results for all the empirical tests, followed by a brief discussion on the overall performance of the \acs{BHH}.

      \item \textbf{Section \ref{sec:results:overview}} provides a brief summary of the chapter.
\end{itemize}